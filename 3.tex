\documentclass[11pt]{llncs}
\usepackage{preamble}

\begin{document}
\title{
Computability Class 2017 - 2018\\
Exercise Set 3}
\date{\today}
\author{Dionysis Zindros\\
    \email{dionyziz@di.uoa.gr}}
\institute{
National and Kapodistrian University of Athens}
\maketitle
\noindent
\makebox[\linewidth]{\small \today}

\thispagestyle{plain}

\section*{Exercise 1}
\subsection*{(a)}
\subsection*{(b)}

\section*{Exercise 2}
\begin{lemma}
Let $\mathcal{L} \in \textsc{RE} \setminus \textsc{REC}$ and some TM $M$ such
that $\mathcal{L}(M) = \mathcal{L}$. Then the set
$S_\mathcal{L} = \{w \in \Sigma^*: M(w)\uparrow\}$ is infinite.
\end{lemma}
\begin{proof}
By contradiction. Suppose $S_\mathcal{L}$ were finite. Then we could construct
a TM $M_2$ which always terminates such that $\mathcal{L}(M_2) = \mathcal{L}$.
The construction of this machine is shown in Algorithm~\ref{alg.ex2}. The
machine first checks if the input word is in the set $S_\mathcal{L}$. Because
the set is finite, it can be encoded in the finite transition function of $M_2$.
In that case, the machine rejects the input. Otherwise, $M_2$ uses universality
to simulate the execution of $M$ on $w$, which is now guaranteed to terminate.
Thus, $\mathcal{L} \in \textsc{REC}$.
\qed
\end{proof}

\begin{figure}[t]
\begin{algorithm}[H]
  \caption{\label{alg.ex2}
      The machine $M_2$ which computes $\mathcal{L}$
      provided that $S_\mathcal{L}$ is finite.
  }
  \begin{algorithmic}[1]
      \Function{$M_2$}{$w$}
        \If{$w \in S_\mathcal{L}$}
            \State\Return{\textsc{False}}
        \EndIf
        \State\Return{$M(w)$}
      \EndFunction
  \end{algorithmic}
\end{algorithm}
\end{figure}

\section*{Exercise 3}
\begin{lemma}
$\mathcal{L} = \{
  \langle M, w \rangle \in \Sigma^*:
  M(w) \text{ visits all non-terminating states of } M
\} \not \in \textsc{REC}$.
\end{lemma}
\begin{proof}
By contradiction. Suppose $\mathcal{L} \in \textsc{REC}$. Then there exists some
TM $M$ that computes $\mathcal{L}$. We now construct an impossible TM $M_3$ as
seen in Algorithm~\ref{alg.ex3}. The machine initially extracts its own encoding
$M_3$ using the Recursion Theorem~\cite{sipser}. Subsequently it utilizes
universality to compute $M(\langle M_3, w \rangle)$. If $M$ returns
\textsc{True}, then $M_3$ terminates without visiting all its non-terminating
states. Otherwise, if $M$ returns $\textsc{False}$, then $M_3$ visits all of its
non-terminating states before terminating. Technically achieving the visitation
of all non-terminating states $Q' = Q \setminus \{Q_\text{accept},
Q_\text{reject}\}$ is not difficult. To do this, the tape is filled with the
special symbols $\sigma_1, \sigma_2, \cdots, \sigma_{|Q'|}$. The head is rewound
to $\sigma_1$ and the active state transitioned to $Q_1$. The transition
function then defines that, upon reading symbol $\sigma_i$ at state $Q_i$, the
machine transitions to state $Q_{i+1}$ and moves to the right. Then it is easy
to see that if $M_3(w)$ visits all non-terminating states, then $M(\langle M_3,
w \rangle)$ will return \textsc{False} and vice versa.
\qed
\end{proof}

\begin{figure}[t]
\begin{algorithm}[H]
  \caption{\label{alg.ex3}
      The impossible machine $M_3$ constructed using $M$.
  }
  \begin{algorithmic}[1]
      \Function{$M_3$}{$w$}
        \Let{\langle M_3 \rangle}{\textsc{SELF}}
        \If{$M(\langle M_3, w \rangle)$}
          \State\Return{\textsc{True}}
        \EndIf
        \State{visit all non-terminating states of $M_3$}
        \State\Return{\textsc{True}}
      \EndFunction
  \end{algorithmic}
\end{algorithm}
\end{figure}

\section*{Exercise 4}
\begin{lemma}
Let $\mathcal{L} = \{
    \langle M \rangle \in \Sigma^*:
    \forall w \in \Sigma^*:
    M(w) \text{ prints 1 before immediately moving left}
\}$. Then $\mathcal{L} \not \in \textsc{REC}$.
\end{lemma}
\begin{proof}
By contradiction. Suppose $\mathcal{L} \in \textsc{REC}$ and let $M$ be its TM.
Then we construct the impossible machine $M_4$ seen in Algorithm~\ref{alg.ex4}.
The machine invokes the Recursion Theorem to determine its own encoding.
Subsequently it calls $M$ using universality passing $\langle M_4 \rangle$. If
$M$ returns \textsc{True}, then $M_4$ terminates. Otherwise it intentionally
prints $1$ and moves left. Care must be taken for the symbol $1$ to not be used
otherwise in this machine's implementation. The contradiction arises from the
fact that if $M_4(\langle M \rangle)$ return \textsc{True}, then $M$ does not
ever print $1$; and if it is \textsc{False}, then $M$ prints $1$ and moves left.
\qed
\end{proof}

\begin{figure}[t]
\begin{algorithm}[H]
  \caption{\label{alg.ex4}
      The impossible machine $M_4$ constructed using $M$.
  }
  \begin{algorithmic}[1]
      \Function{$M_4$}{$w$}
        \Let{\langle M_4 \rangle}{\textsc{SELF}}
        \If{$M(\langle M_4 \rangle)$}
          \State\Return{\textsc{True}}
        \EndIf
        \State{print $1$ and move left}
        \State\Return{\textsc{True}}
      \EndFunction
  \end{algorithmic}
\end{algorithm}
\end{figure}

\section*{Exercise 5}
\begin{lemma}
Let $\mathcal{L} = \{
    \langle M \rangle \in \Sigma^*:
    \forall w \in \Sigma^*:
    M(w) \text{ terminates after an even number of steps}
\}$. Then $\mathcal{L} \not \in \textsc{REC}$.
\end{lemma}
\begin{proof}
By contradiction by reduction from the Halting Problem. Suppose $\mathcal{L} \in
\textsc{REC}$ and let $M_\mathcal{L}$ be its TM. Then we construct an impossible
TM $M_5$ that solves the halting problem as seen in Algorithm~\ref{alg.ex5}.
Specifically, the machine $M_5$ takes as input a machine $\langle M \rangle$
and outputs \textsc{True} if $M$ terminates for all inputs, or \textsc{False} if
there exists some input $w$ such that $M(w)$ does not terminate.
The machine $M_5$ works as follows. First, it uses the function
\textsc{make-even} to transform the input machine $M$ into a machine $M'$ that
behaves identically to $M$, except that it always takes an even number of steps.
The technical details behind this construction are straightforward: Each state
$Q_i$ of $M$ is converted into two states $Q_i^0$, $Q_i^1$ of $M'$ such that
$Q_i^0$ always transitions to $Q_i^1$ and leaves the tape head in position,
while $Q_i^1$ behaves identically to $Q_i^0$. Subsequently, it invokes
universality to compute $M_\mathcal{L}(\langle M' \rangle)$. Because for the
same input $w$, either both $M'(w)$ and $M(w)$ halt or both do not halt,
therefore $M_5$ solves the halting problem for $\langle M \rangle$. \qed
\end{proof}

\begin{figure}[t]
\begin{algorithm}[H]
  \caption{\label{alg.ex5}
      The impossible halting-problem machine $M_5$
      constructed using $M_\mathcal{L}$.
  }
  \begin{algorithmic}[1]
      \Function{$M_5$}{$\langle M \rangle$}
        \Let{\langle M' \rangle}{\textsc{make-even}(\langle M \rangle)}
        \If{$M_\mathcal{L}(\langle M' \rangle)$}
          \State\Return{\textsc{True}}
        \EndIf
        \State\Return{\textsc{False}}
      \EndFunction
  \end{algorithmic}
\end{algorithm}
\end{figure}

\section*{Exercise 6}
\begin{lemma}
Let $\mathcal{L} = \{
  \langle M_1, M_2 \rangle \in \{0, 1\}^*:
  \exists w \in \{0, 1\}^*: \exists n, m \in \mathbb{N}:
  M_1(w)\downarrow^m_{q_\text{yes}} \land
  M_2(w)\downarrow^n_{q_\text{yes}} \land
  n \neq m
\}$. Then $\mathcal{L} \not \in \textsc{REC}$.
\end{lemma}
\begin{proof}
By contradiction by reduction from the Halting Problem. Suppose
$\mathcal{L} \in \textsc{REC}$ and let $M_\mathcal{L}$ be its TM. Then we
construct an impossible TM $M_6$ that solves the Halting Problem. Specifically,
$M_6$ takes as input a TM $\langle M \rangle$ and returns \textsc{True} if $M_6$
terminates with $q_\text{yes}$ for all inputs; or \textsc{False} if there exists
some input $w$ such that $M_6(w)$ does not terminate or terminates with
$q_\text{no}$. The machine is described in Algorithm~\ref{alg.ex6} and works as
follows. First, the machine $M_5$ invokes \textsc{increase-steps} on
$\langle M \rangle$ to obtain $\langle M' \rangle$, a machine that behaves
identically to $M$, except that it has $100$ more states which it goes through
at the beginning of its execution. Then, invoking universality, $M_5$ calls
$M_\mathcal{L}$ with parameters $M', M_\text{trivial}$, where $M_\text{trivial}$
is a machine which always terminates with $q_\text{yes}$ in one step.

To see why $M_5$ solves the halting problem, observe that it will return
\textsc{True} only if $M'$ terminates with $q_\text{yes}$ for some input in a
different number of steps than $M_\text{trivial}$. But if $M'$ terminates at
all, then it will terminate in a larger number of steps than $M_\text{trivial}$,
since $M_\text{trivial}$ always takes 1 step, while $M'$ always takes at least
$100$ steps. But $M'$ terminates with $q_\text{yes}$ iff $M$ terminates
with $q_\text{yes}$. Therefore $M_5$ solves the halting problem.
\qed
\end{proof}

\begin{figure}[t]
\begin{algorithm}[H]
  \caption{\label{alg.ex6}
      The impossible halting-problem machine $M_6$
      constructed using $M_\mathcal{L}$.
  }
  \begin{algorithmic}[1]
      \Function{$M_5$}{$\langle M \rangle$}
        \Let{\langle M' \rangle}{\textsc{increase-steps}(\langle M \rangle)}
        \If{$M_\mathcal{L}(\langle M', M_\text{trivial} \rangle)$}
          \State\Return{\textsc{True}}
        \EndIf
        \State\Return{\textsc{False}}
      \EndFunction
  \end{algorithmic}
\end{algorithm}
\end{figure}

\section*{Exercise 7}
\begin{lemma}
$\forall \mathcal{L} \in \textsc{RE}:
\mathcal{L} \leq_m
\{\langle M \in \Sigma^* \rangle: M(\langle M \rangle)\downarrow\}$
\end{lemma}
\begin{proof}
Let $\mathcal{L}$ be an arbitrary language in \textsc{RE} and let its recognizer
TM be $M$. We will now construct a computable reduction $f_7$ as shown in
Algorithm~\ref{alg.ex7}. The computable function $f_7$ immediately returns the
description of a machine $M_7$ which it does not execute. The machine $M_7$
ignores its input and has
the constant $w$ encoded in it. Using universality, it invokes $M$ with input
$w$. The reduction is correct because, if $w \in \mathcal{L}$, then $M_7$ will
terminate for all inputs, and thus
$M_7 \in \{\langle M \in \Sigma^* \rangle:
M(\langle M \rangle)\downarrow\}$. Otherwise, if $w \not \in \mathcal{L}$, then
$M_7$ will not terminate for any input, and thus
$M_7 \not \in \{\langle M \in \Sigma^* \rangle: M(\langle M
\rangle)\downarrow\}$.
\qed
\end{proof}

\begin{figure}[t]
\begin{algorithm}[H]
  \caption{\label{alg.ex7}
      A reduction for an \textsc{RE} language.
  }
  \begin{algorithmic}[1]
      \Function{$f_7$}{$w$}
        \State\Return{The machine described as follows:}
        \Function{$M_7$}{$\_$}
          \If{$M(w)\downarrow_{q_\text{yes}}$}
              \State\Return{\textsc{True}}
          \Else
              \State\Return{\textsc{False}}
          \EndIf
        \EndFunction
      \EndFunction
  \end{algorithmic}
\end{algorithm}
\end{figure}


\section*{Exercise 8}
\begin{lemma}
$\mathcal{L} \in \textsc{REC} \Leftrightarrow \mathcal{L} \leq_m 0^*1^*$.
\end{lemma}
\begin{proof}
For the forward direction. Suppose $\mathcal{L} \in \textsc{REC}$ and its TM is
$M$. Then it is straightforward to transform $M$ to compute the reduction
function $f(w)$ which outputs $\epsilon$ if $w \in \mathcal{L}$ or $10$
as seen in Algorithm~\ref{alg.ex8a}. $\epsilon$ matches the given regular
expression, while $10$ does not.

For the reverse direction. Suppose there exists some computable reduction $f$.
Then we can easily construct a TM $M$ which computes $\mathcal{L}$ as seen in
Algorithm~\ref{alg.ex8b}. The machine first invokes the reduction function $f$
to transform the input $w$ into a new word $w'$ which matches the regular
expression $0^*1^*$ iff $w \in \mathcal{L}$. The Turing Machine then checks if
$w'$ matches the regular expression, which we denote with the $\sim$ operator.
Regular expression matching is computable due to regularity.
\qed
\end{proof}

\begin{figure}[t]
\begin{algorithm}[H]
  \caption{\label{alg.ex8a}
      From a Turing Machine $M$ to a computable function $f_8$.
  }
  \begin{algorithmic}[1]
      \Function{$f_8$}{$w$}
        \If{$M(w)\downarrow_{q_\text{yes}}$}
            \State\Return{$\epsilon$}
        \Else
            \State\Return{$10$}
        \EndIf
      \EndFunction
  \end{algorithmic}
\end{algorithm}
\end{figure}

\begin{figure}[t]
\begin{algorithm}[H]
  \caption{\label{alg.ex8b}
      From a computable function $f$ to a Turing Machine $M_8$.
  }
  \begin{algorithmic}[1]
      \Function{$M_8$}{$w$}
        \Let{w'}{f(w)}
        \If{$w' \sim 0^*1^*$}
            \State\Return{\textsc{True}}
        \Else
            \State\Return{\textsc{False}}
        \EndIf
      \EndFunction
  \end{algorithmic}
\end{algorithm}
\end{figure}


\section*{Exercise 9}
\begin{lemma}
Let
$\mathcal{L} = \{
    \langle M \rangle \in \Sigma^*:
    \exists w \in \Sigma^*:
    |w| = 1871 \land
    w \in \mathcal{L}(M)
\}$. Then $\mathcal{L} \in \textsc{RE}$, but
$\overline{\mathcal{L}} \not \in \textsc{RE}$.
\end{lemma}
\begin{proof}
For the inclusion result, we construct an enumerator machine $M_9$ shown in
Algorithm~\ref{alg.ex9}. The machine loops through all pairs $(M, i)$ in
a dove-tailing fashion, where $M$ is a TM and $i$ is an integer number of steps.
Then, for all $1871$-sized inputs $w$, it checks if the machine $M$ outputs
$w$ in exactly $i$ steps, and, if so, emits it.

For the exclusion result, suppose for contradiction that $\overline{\mathcal{L}}
\in \textsc{RE}$. Then $\mathcal{L} \in \textsc{REC}$. But
$\textsc{C} = \{
    \mathcal{L}:
    \exists w \in \Sigma^*: |w| = 1871 \land w \in \mathcal{L}
\}$
is a non-trivial complexity class and $\mathcal{L}$ contains all the TM which
compute languages in \textsc{C}, and thus contradicts
Rice's Theorem.
\qed
\end{proof}

\begin{figure}[t]
\begin{algorithm}[H]
  \caption{\label{alg.ex9}
      The machine $M_9$ which enumerates all machines
      outputting words of a specific size.
  }
  \begin{algorithmic}[1]
      \Function{$M_9$}{}
        \For{dove-tail $(\langle M \rangle, i) \in \mathbb{N}^2$}
            \For{$w \in \Sigma^{1871}$}
                \If{$M(w)\downarrow^i_{q_\text{yes}}$}
                    \State{\textsf{Emit }}{M(w)}
                \EndIf
            \EndFor
        \EndFor
      \EndFunction
  \end{algorithmic}
\end{algorithm}
\end{figure}

\section*{Exercise 10}

\bibliography{bibliography}
\end{document}
