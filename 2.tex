\documentclass[11pt]{llncs}
\usepackage{preamble}

\begin{document}
\title{
Computability Class 2017 - 2018\\
Exercise Set 2}
\date{\today}
\author{Dionysis Zindros\\
    \email{dionyziz@di.uoa.gr}}
\institute{
National and Kapodistrian University of Athens}
\maketitle
\noindent
\makebox[\linewidth]{\small \today}

\thispagestyle{plain}

\section*{Exercise 1}
\begin{lemma}
All primitive recursive functions are total.
\end{lemma}
\begin{proof}
By induction on the recursive definition of primitive recursive functions. The
\textit{constant}, \textit{successor}, and \textit{projection} functions are trivially total. For
\textit{composition}, suppose by induction that $f$ and $g_1, \cdots, g_k$ are total,
then let:

\[
  h(x_1, \cdots, x_m) = f(g_1(x_1, \cdots, x_m), \cdots, g_k(x_1, \cdots,
x_m))
\]

Clearly $h$ is total. For \textit{primitive recursion}, suppose by induction that $f$ and
$g$ are total. Then let:

\[
  h(S(y), x_1, \cdots, x_k) = g(y, h(y, x_1, \cdots, x_k), x_1, \cdots, x_k)
\]

For contradiction fix the smallest $S(y)$ such that $h$ is not
total. But clearly the right-hand side is total.
\qed
\end{proof}

\section*{Exercise 2}
\subsection*{(a)}
\begin{lemma}
If $g$ is primitive recursive, then the function \textsf{sum} is primitive
recursive:

\[
\textsf{sum}(z, x) = \sum_{i = 0}^z g(i, x)
\]
\end{lemma}
\begin{proof}
We can construct \textsf{sum} as follows:

\begin{align*}
  \textsf{sum}(0, x) &= g(0, x)\\
  \textsf{sum}(i, x) &= \textsf{sum}(i - 1, x) + g(i, x)
\end{align*}

The last part can be written formally:

\[
  \textsf{sum}(S(i), x)
  = \textsf{sum}(i, x) + g(i, x) = f(i, \textsf{sum}(i, x), x)
\]

By the fact that $g$ and $+$ are primitive recursive, $f$ is also primitive
recursive and using primitive recursion on the first argument of \textsf{sum},
we conclude that \textsf{sum} is primitive recursive.
\qed
\end{proof}
\begin{lemma}
If $g$ is primitive recursive, then the function \textsf{prod} is primitive
recursive:

\[
  \textsf{prod} = \prod_{i=0}^z g(i, x)
\]
\end{lemma}
\begin{proof}
The construction for \textsf{prod} is similar:

\begin{align*}
  \textsf{prod}(0, x) &= g(0, x)\\
  \textsf{prod}(S(i), x) &= \textsf{prod}(i, x) * g(i, x)
\end{align*}
\qed
\end{proof}

\subsection*{(b)}

\begin{lemma}
The function $(\mu\>i \leq z)[(i, x) \in R]$ is primitive recursive.
\end{lemma}
\begin{proof}
Assume $(\mu\>i \leq z)[(i, x) \in R]$ returns $z + 1$ if
$\not\exists i \leq z: (i, x) \in R$.

We define $f$ using the primitive recursion constructs:

\begin{align*}
  f(0, x) &= 0\\
  f(i + 1, x) &= \begin{cases}
             f(i, x) \text{, if } f(i, x) \neq i\\
             i + 1 \text{, if } (i + 1, x) \in R\\
             i + 2 \text{, otherwise}
            \end{cases}
\end{align*}

The above can be implemented using primitive recursion because \textsf{if} is
primitive recursive.
\qed
\end{proof}

\section*{Exercise 3}
\begin{lemma}
Let $g_1, g_2, h_1, h_2$ be primitive recursive functions. Let $f_1, f_2$ be
defined as follows:

\begin{align*}
  f_1(0, x)     &= g_1(x)\\
  f_1(y + 1, x) &= h_1(f_1(y, x), f_2(y, x), y, x)\\
  \\
  f_2(0, x)     &= g_2(x)\\
  f_2(y + 1, x) &= h_2(f_1(y, x), f_2(y, x), y, x)\\
\end{align*}

Then $f_1, f_2$ are primitive recursive.
\end{lemma}
\begin{proof}
We create a function $f$ representing the pair $f_1, f_2$. Pairs of numbers can
be G\"odel encoded and decoded with primitive recursive functions. We denote
this encoding with $f_1\|f_2$ and the decoding with $(f_1\|f_2)_1$ and
$(f_1\|f_2)_2$ respectively.

Then $f$ is primitive recursive:
\begin{align*}
  f(0, x) &= g_1(x)\|g_2(x)\\
  f(y + 1, x) &= h_1'(f(y, x), y, x)\|h_2'(f(y, x), y, x)
\end{align*}

Note that we needed to define the primitive recursive functions $h_1', h_2'$
that first decode $f(y, x)_1$ and $f(y, x)_2$ respectively prior to applying
$h_1$ or $h_2$ to them.

But $f_1(y, x) = f(y, x)_1$ and $f_2(y, x) = f(y, x)_2$ and so they are
primitive recursive.
\qed
\end{proof}

\section*{Exercise 4}
\begin{lemma}
  Let $g, h, \tau$ be functions. Then let $f$ be defined as the solution to the
  following system of equations:

  \begin{align*}
    f(0, y) &= g(y)\\
    f(x + 1, y) &= h(f(x, \tau(x, y)), x, y)
  \end{align*}

  Then $f$ exists and, if $g, h, \tau$ are primitive recursive, so is $f$.
\end{lemma}
\begin{proof}
$f$ is well-defined because the recursive step always reduces $x$. Because the
reduction is always by one, it will eventually reach $0$.

For the primitive recursion part, use the notation above and define $f'$ as
follows:

\begin{align*}
  f'(0, y) &= g(y)\|\tau(0, y)\\
  f'(x + 1, y) &= h(f'(x, y)_{x+1}, x, y)\|\tau(0, y)\|\cdots\|\tau(x + 1, y)
\end{align*}

Essentially, for a fixed $y$, $f'(x, y)$ collects all the values of $\tau(i, y)$
for $i \leq x$ and G\"odel encodes them together. Because the number of $\tau$
values that need to be calculated and encoded is bounded (by $x + 1$), and
because $\tau$ is primitive recursive, therefore $f'$ is also primitive
recursive.

But then $f(x, y) = f'(x, y)_1$ is also primitive recursive.
\qed
\end{proof}

\section*{Exercise 5}
Give an example of a $\mu$-recursive function which is not primitive recursive.

\paragraph{Solution.}

The Ackermann function:

\[
  A(m, n) = \begin{cases}
    n + 1 \text{, if } m = 0\\
    A(m - 1, 1 \text{, if } m > 0 \text{ and } n = 0\\
    A(m - 1, A(m, n - 1)) \text{, otherwise}
  \end{cases}
\]

\section*{Exercise 6}

Find a grammar for the language $\mathcal{L} = \{ ww: w \in \{0, 1\}^* \}$.

\paragraph{Solution.}

\begin{align*}
  S  &\rightarrow AB\\
  A  &\rightarrow A0P | A1Q | \epsilon\\
  P0 &\rightarrow 0P\\
  P1 &\rightarrow 1P\\
  PB &\rightarrow B0\\
  Q0 &\rightarrow 0Q\\
  Q1 &\rightarrow 1Q\\
  QB &\rightarrow Q1
\end{align*}

\section*{Exercise 7}

\begin{lemma}
$\mathcal{L} = \{a^n b^n c^n: n \geq 1\}$ is context-sensitive.
\end{lemma}
\begin{proof}
By construction. $\mathcal{L}$ is generated by the following non-contracting
grammar, and therefore it is context-sensitive:

\begin{align*}
  S  &\rightarrow ABC\\
  A  &\rightarrow AaP | a\\
  B  &\rightarrow b\\
  C  &\rightarrow c\\
  Pa &\rightarrow aP\\
  PB &\rightarrow BbP\\
  Pb &\rightarrow bP\\
  PC &\rightarrow Cc
\end{align*}
\qed
\end{proof}

\section*{Exercise 8}
\subsection*{(a)}
Find a grammar that generates the language
$\mathcal{L} = \{1^{2^i}: i \geq 1\}$.

\paragraph{Solution.}

\begin{align*}
  S  &\rightarrow D1Z\\
  D  &\rightarrow DD | \epsilon\\
  D1 &\rightarrow 11P\\
  P1 &\rightarrow 11P\\
  PZ &\rightarrow Z\\
  Z  &\rightarrow \epsilon\\
\end{align*}

\subsection*{(b)}
\begin{lemma}
$\mathcal{L} = \{1^{2^i}: i \geq 1\}$ is context-sensitive.
\end{lemma}
\begin{proof}
By construction. $\mathcal{L}$ is generated by the following
non-contracting grammar and is therefore context-sensitive:

\begin{align*}
  S    &\rightarrow 1 | 11 | 1111 | D1Z\\
  D    &\rightarrow DD\\
  D1   &\rightarrow 11P\\
  P1   &\rightarrow 11P\\
  D1Z  &\rightarrow 11Z\\
  D11Z &\rightarrow 1111\\
  P1Z  &\rightarrow 11Z\\
  P11Z &\rightarrow 1111
\end{align*}
\qed
\end{proof}

\section*{Exercise 9}
Show that each context-sensitive grammar $G$ generates a language which can be
generated by a grammar $G'$ with rules of the form $uAv \rightarrow uwv$ where
$A \in V \setminus \Sigma$ and $u, v, w \in V^*$ with $w \neq \epsilon$.

\paragraph{Solution.}
This is clearly true, as the form of the rules for $G'$ are exactly the same as
the rules constricting a context-sensitive grammar \cite{chomsky}.

Furthermore, removing the restriction $w \neq \epsilon$ produces a class
of grammars which generate all the languages in $\textsc{RE}$.

\section*{Exercise 10}
\begin{lemma}
  $\mathcal{L}_{LBA} = \{\langle M, w \rangle: M \text{ is an LBA and } M(w)\downarrow q_\text{yes}\} \in \textsc{REC}$.
\end{lemma}
\begin{proof}
By construction. Let $T$ be a Turing Machine which decides $\mathcal{L}_{LBA}$.
It works as follows: On input $\langle M, w \rangle$, $T$ decodes $M$ into its
alphabet $\Sigma$ and its state set $Q$, then simulates $M$ with input $w$ for
$\alpha = \Sigma^{|w|} |Q| |w|$ steps. If at any time during the simulation $M$
has accepted, then $M$ accepts. If $M$ has rejected, then $T$ rejects. If $T$
has not reached a halting state, then $T$ rejects. Because $T$ can only reuse
the tape space occupied by the input, its possible configurations are up to
$\alpha$ and, hence, since it did not halt within $\alpha$ simulation steps, it
necessarily loops.
\qed
\end{proof}

\bibliography{bibliography}
\end{document}
