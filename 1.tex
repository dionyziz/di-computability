\documentclass[11pt]{llncs}
\usepackage{preamble}

\begin{document}
\title{
Computability Class 2017 - 2018\\
Exercise Set 1}
\date{\today}
\author{Dionysis Zindros\\
    \email{dionyziz@di.uoa.gr}}
\institute{
National and Kapodistrian University of Athens}
\maketitle
\noindent
\makebox[\linewidth]{\small \today}

\thispagestyle{plain}

\section*{Exercise 1}
Let $f: \Sigma^* \rightarrow \Sigma^*$ be a one-to-one onto computable function.
Show that $f^{-1}$ is computable.

\paragraph{Solution.}

$f^{-1}$ is clearly computable when $f$ is total through direct brute force.
To show that $f^{-1}$ is computable in the case when $f$ is partial, we
will devise an algorithm which finds the inverse in a finite number of
steps, provided that it exists. This algorithm is shown in
Algorithm~\ref{alg.1}. Intuitively, the algorithm attempts to run all steps of
all inputs ``in parallel'' without encountering an infinite loop.

\begin{figure}[t]
\begin{algorithm}[H]
  \caption{\label{alg.1}
      The generic function inversion algorithm
      $\textsf{inverse}_M$
      parameterised by a machine $M$ which computes
      a partial function $f$}
  \begin{algorithmic}[1]
      \Function{{\sf inverse}$_M$}{$x$}
        \Let{\ell}{0}
        \While{\textsc{True}}
            \Let{\ell}{\ell + 1}
            \State{Initialise machine $M$ for input $\ell$}
            \Let{i}{1}
            \For{$y = \ell \text{ down to } 1$}
                \State{Simulate machine $M$ with input $y$ for exactly $i$ steps}
                \If{$M$ \text{has terminated with output} $x$}
                    \State\Return{$y$}
                \EndIf
                \Let{i}{i + 1}
            \EndFor
        \EndWhile
      \EndFunction
  \end{algorithmic}
\end{algorithm}
\end{figure}

The algorithm is parameterised by a Turing Machine $M$ which computes
the partial function $f$ and works as follows. It attempts to invert the
function by incrementing a variable $\ell$. At each loop iteration, $\ell$
denotes the maximum input as well as the maximum number of steps allowed.

\begin{lemma}
If $\textsf{inverse}_M$ is executed with an input $x$ such that there exists
a $y$ for which $f(x) = y$, then $\textsf{inverse}_M$ will terminate with output
$y$.
\end{lemma}
\begin{proof}
Because $f$ has a value for input $x$, therefore $M$ will terminate with input
$x$. Assume this termination occurs after $i$ steps.

Then, the main for loop of the $\textsf{inverse}_M$ function will terminate in
the loop at the iteration $\ell = i + y$. To see this, observe that every
iteration of the \textit{for} loop is executed in constant time.
\qed
\end{proof}

\section*{Exercise 2}
Let $\mathcal{L} \in \Sigma^*$. Show that $\mathcal{L} \in \textsc{RE}$ if and
only if $\mathcal{L} = \{x \in \Sigma^*|\exists y \in \Sigma^*:
\langle x, y \rangle \in B\}$, where $B \in \textsc{REC}$.

\paragraph{Solution.}

We will let $y$ be the bound in the number of execution steps of
the Turing Machine. This will be utilized as a certificate of termination.

\paragraph{For the forward direction.} Assume $\mathcal{L} \in \textsc{RE}$. Then
there must exist a Turing Machine $M$ which enumerates it. Now construct the
following set:

\[
  B = \{\langle x, y \rangle: M(x) \text{ terminates in } y \text{ steps}\}
\]

\begin{lemma}
$\mathcal{L} = \{x: \exists y: \langle x, y \rangle \in B\}$
\end{lemma}
\begin{proof}
By definition, if $x$ is recognised by $M$ then $M$ must terminate in a finite
number of steps.
\qed
\end{proof}
\begin{lemma}
$B \in \textsc{REC}$
\end{lemma}
\begin{proof}
We will prove this by constructing a Turing Machine $M'$ which computes $B$.
This is done directly as follows: If an input $\langle x, y \rangle$ is given to
$M'$, then $M'$ simulates $M$ for exactly $y$ steps. If $M$ has reached an
acceptance state after $y$ steps, then $M'$ outputs \textit{yes}. Otherwise,
$M'$ outputs \textit{no}.
\qed
\end{proof}

\paragraph{For the inverse direction.} Let  $\mathcal{L} = \{x: \exists y:
\langle x, y \rangle \in B\}$ for some $B \in \textsc{REC}$.

\begin{lemma}
$\mathcal{L} \in \textsc{RE}$
\end{lemma}
\begin{proof}
Assume $B$ is computed by some Turing Machine $M'$. We will prove the lemma
constructively by building a Turing Machine $M$ for $\mathcal{L}$. The
construction is straightforward: Given $x$, the machine loops to try all $y \in
\mathbb{N}$. For each of these $y$, it simulates $M'$ with input $\langle x, y
\rangle$. Because $B \in \textsc{REC}$, therefore $M'$ will terminate in a
finite number of steps for all of these inputs. Therefore, if a $y$ exists such
that $\langle x, y \rangle \in B$, it will eventually be found. In that case,
$M$ terminates and outputs \textit{yes}. Otherwise, $M$ does not terminate.
\qed
\end{proof}

\section*{Exercise 3}
Let $\mathcal{L} \subseteq \Sigma^*$. We define:

\begin{align*}
\mathcal{L}^0 &= \{\epsilon\}\\
\mathcal{L}^1 &= \mathcal{L}\\
\mathcal{L}^n &= \{w_1 \circ w_2 \circ \cdots \circ w_n \in \Sigma^*
  : w_1, w_2, \cdots, w_n \in \mathcal{L}\}\\
\mathcal{L}^* &= \bigcup_{i \in \mathbb{N}} \mathcal{L}^i
\end{align*}

Show that if $\mathcal{L} \in \textsc{RE}$ then $\mathcal{L}^* \in \textsc{RE}$.

\paragraph{Solution.}

\begin{figure}[t]
\begin{algorithm}[H]
  \caption{\label{alg.3}
      The string partition Turing Machine parameterised by
      a substring recogniser machine $M$
  }
  \begin{algorithmic}[1]
      \Function{{\sf partition}$_M$}{$w$}
        \State{Non-deterministically split $w$ into partition $w_1, w_2, \cdots w_n$}
        \For{$i = 1 \text{ to } n$}
            \If{$M(w_i) = \textsc{reject}$}
                \State\Return{\sc reject}
            \EndIf
        \EndFor
        \State\Return{\sc accept}
      \EndFunction
  \end{algorithmic}
\end{algorithm}
\end{figure}

\begin{lemma}
$\mathcal{L} \in \textsc{RE} \Rightarrow \mathcal{L}^* \in \textsc{RE}$
\end{lemma}
\begin{proof}
Let $\mathcal{L}$ be recognisable by some Turing Machine $M$. We will prove this
lemma constructively by building a Turing Machine $M^*$ for $\mathcal{L}^*$. The
machine $M^*$ is shown in Algorithm~\ref{alg.3}. It takes $w$ as its input.
Non-deterministically, it splits $w$ into all possible substrings $w_1, w_2,
\cdots, w_n$ for all $n = 0 \cdots |w| - 1$. For each of these possible
partitions, it simulates the execution of $M$ with input $w_i$. If any of the
non-deterministic partitions is successful, none of the executions of $M$ will
reject, but all will terminate. Therefore, the machine will accept the input.
\qed
\end{proof}

\section*{Exercise 4}

\subsection*{(a) Recursive case}
Show that every recursive language is the union of two disjoint infinite
recursive languages.

\paragraph{Solution.}

Let $\mathcal{L} \in \textsc{REC}$. We will construct $\mathcal{L}_1,
\mathcal{L}_2$ such that $\mathcal{L}_1 \cup \mathcal{L}_2 = \mathcal{L}$ and
$\mathcal{L}_1 \cap \mathcal{L}_2 = \emptyset$.

We sort $\mathcal{L}$ in increasing order and let $\mathcal{L}_1$ be those items
of $\mathcal{L}$ with odd index and $\mathcal{L}_2$ to be those with even index.
Trivially, $\mathcal{L}_1 \cap \mathcal{L}_2 = \emptyset$ and they are both
infinite due to the infinity of $\mathcal{L}$.

\begin{figure}[t]
\begin{algorithm}[H]
  \caption{\label{alg.4a}
      Splitting a \textsc{REC} language $\mathcal{L}$ into disjoint
      partitions. The algorithm is parameterised by a Turing Machine $M$ for
      $\mathcal{L}$.
  }
  \begin{algorithmic}[1]
      \Function{{\sf even}$^{\sc REC}_M$}{$x$}
          \If{$M(x) = \textsc{reject}$}
              \State\Return{\textsc{reject}}
          \EndIf
          \Let{cnt}{0}
          \For{$i = 1 \text{ to } x$}
              \If{$M(i) = \textsc{accept}$}
                  \Let{cnt}{cnt + 1}
              \EndIf
          \EndFor
          \If{$cnt\ \%\ 2 = 0$}
              \State\Return{\sc accept}
          \EndIf
          \State\Return{\sc reject}
      \EndFunction
  \end{algorithmic}
\end{algorithm}
\end{figure}

\begin{lemma}
$\mathcal{L}_1, \mathcal{L}_2 \in \textsc{REC}$
\end{lemma}
\begin{proof}
We will show it for $\mathcal{L}_2$. We do this by construction. Let $M$ be a
Turing Machine that recognises $M$. The construction is shown in
Algorithm~\ref{alg.4a}. The new machine, $\textsf{even}^{\sc REC}_M$ machine
takes an input $x$. If $M$ rejects $x$, so does $\textsf{even}^{\sc REC}_M$.
Otherwise, it simulates the execution of $M$, counting the problem instances up
to $x$ in which $M$ accepted. It only accepts if the count is even.

The construction for $\mathcal{L}_1$ is analogous.
\qed
\end{proof}

\subsection*{(b) Recursively enumerable case}
Show that every recursively enumerable language is the union of two disjoint
infinite recursively enumerable languages.

\paragraph{Solution.}

Let $\mathcal{L} \in \textsc{RE}$. As such, there is an enumerator $M$. Assume
without loss of generality that this enumerator enumerates the language without
repetitions. Again we will construct $\mathcal{L}_1, \mathcal{L}_2$ such that
$\mathcal{L}_1 \cup \mathcal{L}_2 = \mathcal{L}$ and $\mathcal{L}_1 \cap
\mathcal{L}_2 = \emptyset$.

The enumerator gives a direct ordering of the elements of the language. Again
we will split the language into even- and odd-indexed items. As before, clearly
these will partition the language into two disjoint sets.

To prove that $\mathcal{L}_1, \mathcal{L}_2$ are in $\sc RE$, it suffices to
show that there is an enumerator for them. For the both cases, the construction
is analogous to the $\textsc{REC}$ case. For example, it is not hard to see that
an enumerator $M'$ for the even case can be constructed by having it simulate
$M$ and only output the even-indexed items.

\section*{Exercise 5}
Let $f: \mathbb{N} \rightarrow \mathbb{N}$ be:

\[
    f(x) = \begin{cases}
        1 \text{, if } \exists w \in \Sigma^*: M_x(w) \downarrow\\
        \uparrow \text{ otherwise}
    \end{cases}
\]

Show that $f$ is computable.

\paragraph{Solution.}

We prove the statement constructively. To compute $f$, we construct a machine
$M$ which outputs $1$ if $\exists w \in \Sigma^*: M_x(w)\downarrow$ by
simulating $M_x(w)$ for all inputs $w$ in a dove-tailing fashion. If
$M_x(w)\uparrow$, then our machine does not halt, corresponding to the
partiality of $f$. The exact construction is presented in Algorithm~\ref{alg.5}.

\begin{figure}[t]
\begin{algorithm}[H]
  \caption{\label{alg.5}
      A machine that recognises whether the $x^{\text{th}}$ Turing Machine
      halts with some input.
  }
  \begin{algorithmic}[1]
      \Function{{\sf halt-recogniser}}{$x$}
          \Let{\ell}{0}
          \While{\textsc{True}}
              \Let{\ell}{\ell + 1}
              \State{Initialise machine $M_x$ for input $\ell$}
              \Let{i}{1}
              \For{$w = \ell \text{ down to } 1$}
                  \State{Simulate machine $M$ with input $w$ for exactly $i$
                  steps}
                  \If{$M$ \text{has terminated}}
                      \State\Return{$1$}
                  \EndIf
                  \Let{i}{i + 1}
              \EndFor
          \EndWhile
      \EndFunction
  \end{algorithmic}
\end{algorithm}
\end{figure}

\section*{Exercise 6}
Let $f: \mathbb{N} \rightarrow \mathbb{N}$ be:

\[
    f(x) = \begin{cases}
        x \text{, if } |\mathcal{L}(M_x)| \geq x\\
        \uparrow \text{ otherwise}
    \end{cases}
\]

Show that $f$ is computable.

\paragraph{Solution.}

Again we will prove this constructively. The construction is almost identical
to the previous case. However, the machine must be modified to maintain a set
of inputs $w$ that have been recognised by $M_x$. Once the set grows beyond the
size $|s| \geq x$, the machine can return $x$. The exact construction is shown
in Algorithm~\ref{alg.6}.

\begin{figure}[t]
\begin{algorithm}[H]
  \caption{\label{alg.6}
      A machine that recognises whether the $x^{\text{th}}$ Turing Machine
      produces a language with at least $x$ elements.
  }
  \begin{algorithmic}[1]
      \Function{{\sf language-counter}}{$x$}
          \Let{\ell}{0}
          \Let{s}{\emptyset}
          \While{\textsc{True}}
              \Let{\ell}{\ell + 1}
              \State{Initialise machine $M_x$ for input $\ell$}
              \Let{i}{1}
              \For{$w = \ell \text{ down to } 1$}
                  \State{Simulate machine $M_x$ with input $w$ for exactly $i$
                  steps}
                  \If{$M_x$ \text{has terminated}}
                      \Let{s}{s \cup \{w\}}
                      \If{$|s| \geq x$}
                          \State\Return{$x$}
                      \EndIf
                  \EndIf
                  \Let{i}{i + 1}
              \EndFor
          \EndWhile
      \EndFunction
  \end{algorithmic}
\end{algorithm}
\end{figure}

\section*{Exercise 7}

Let $\mathcal{L}$ be defined as follows:

\begin{align*}
\mathcal{L} = \{\langle M, w, n \rangle \in \Sigma^*:\ &n \neq 1
                \text{ and during the computation of } M(w)\\
                &\text{the head visits the } n^{\text{th}}
                \text{ position of the tape}\}
\end{align*}

Determine whether $\mathcal{L} \in \textsc{REC}$.

\paragraph{Solution.}

We show that $\mathcal{L} \in \textsc{REC}$ by construction. The exact
construction is shown in Algorithm~\ref{alg.7}. Our machine simulates $M$ until
it either visits a tape position past $n$ or returns to a previously known
configuration\footnote{Configurations are implied to be represented by some
G\"odel encoding.}, in which case we can deduce it will never halt.

\begin{figure}[t]
\begin{algorithm}[H]
  \caption{\label{alg.7}
      The machine \textsf{tape-visited} that determines whether a particular
      tape position $n$ is visited by another machine $M$ with input $w$.
  }
  \begin{algorithmic}[1]
      \Function{{\sf tape-visited}}{$\langle M, w, n \rangle$}
          \Let{s}{\emptyset}
          \State{Initialise machine $M$ for input $w$}
          \Let{c}{\text{initial configuration of } M(w)}
          \Let{s}{s \cup \{c\}}
          \While{\textsc{True}}
              \State{Simulate machine $M(w)$ for one more step}
              \Let{c}{\text{next configuration of } M(w)}
              \If{$M(w)$ visits tape position $n$}
                  \State\Return{\sc accept}
              \EndIf
              \If{$c \in s$}
                  \State\Return{\sc reject}
                  \Comment{Looping machine}
              \EndIf
              \Let{s}{s \cup \{c\}}
              \If{$M(w)$ has terminated}
                  \State\Return{\sc reject}
              \EndIf
              \Let{i}{i + 1}
          \EndWhile
      \EndFunction
  \end{algorithmic}
\end{algorithm}
\end{figure}

\begin{lemma}
\textsc{tape-visited} halts for all inputs and computes $\mathcal{L}$.
\end{lemma}
\begin{proof}
First, we note that \textsc{tape-visited} accepts if and only if $M(w)$ uses
tape position $n$. Indeed, the accepting state is only reached when the
simulated machine $M$ goes through a configuration which visits position $n$.
Furthermore, because the simulation is done in single-step execution, a machine
which visits the $n^{\text{th}}$ position of the tape must go through such a
configuration.

Next, we note that if $M(w)$ terminates without visiting tape position $n$, then
\textsc{tape-visited} rejects. This follows directly from the termination check.
Conversely, if \textsc{tape-visited} rejects, then this must only occur if the
tape position $n$ is never visited. To see this, note that rejection occurs if
the machine loops (and so a configuration $c$ previously attained is visited
with $c \in s$), or $M(w)$ terminates without reaching $n$.

Finally, we notice that \textsc{tape-visited} always terminates, even if $M(w)$
does not terminate. To see this, assume for contradiction that
\textsc{tape-visited} did not terminate, and observe that the only way for
this to happen is if $M(w)$ does not terminate by always reaching a unique new
configuration $c$. But the set $s$ of all machine configurations for a Turing
Machine $M$ that runs with input $w$ and does not ever go past $n$ is finite.
Indeed, $|s| = |\Sigma|^n |Q| n$.
\qed
\end{proof}

\section*{Exercise 8}
Let $\mathcal{L} \subseteq \Sigma^*$. Show that if $\mathcal{L} \in \textsc{RE}$
then $\mathcal{L}_p = \{w \in \mathcal{L}: w = w^R\} \in \textsc{RE}$.

\paragraph{Solution.}

Again we will work constructively. Let $\mathcal{L} \in \textsc{RE}$ and $M$ be
an enumerator for $\mathcal{L}$. We will construct an enumerator $M_p$ for
$\mathcal{L}_p$. The enumerator construction is straightforward: The enumerator
$M_p$ simulates the execution of $M$. Whenever $M$ generates some output $w$,
then $M_p$ simply checks if $w = w^R$. If yes, then $M_p$ generates $w$.
Otherwise, $M_p$ skips this particular output $w$ and continues with the
simulation.

\section*{Exercise 9}

Let $A, B, C \subseteq \Sigma^*$. We call $C$ a distinguisher of $A$ and $B$ if
$A \subseteq C$ and $B \subseteq \overline C$. Let $A, B \subseteq \Sigma^*$
with $A \cap B = \emptyset$ and $\overline A, \overline B \in \textsc{RE}$.
Show that there is some $C \in \textsc{REC}$ which is a distinguisher of $A, B$.

\paragraph{Solution.}

We show by construction that there is a $C \in \textsc{REC}$ which separates $A$
and $B$. Because $\overline A, \overline B \in \textsc{RE}$, there are Turing
Machines $N$ and $M$ which recognise $\overline A$ and $\overline B$
respectively. To construct a Turing Machine $\textsf{distinguish}_{N,M}$ for
some $C$ we follow the construction shown in Algorithm~\ref{alg.9}.

\begin{figure}[t]
\begin{algorithm}[H]
  \caption{\label{alg.9}
      A machine that computes a distinguisher $C$ of two languages $A, B \in
      \text{co-}\textsc{RE}$. It is parameterised by their Turing Machines $N,
      M$.
  }
  \begin{algorithmic}[1]
      \Function{{\sf distinguish}$_{N,M}$}{$w$}
          \State{Initialise machines $N(w), M(w)$}
          \Let{i}{1}
          \While{\textsc{True}}
              \State{Simulate one more step of $N(w)$}
              \If{$N(w)$ has halted with \textsc{accept}}
                  \Comment{$w \not\in A$}
                  \State\Return{\textsc{reject}}
              \EndIf
              \State{Simulate one more step of $M(w)$}
              \If{$M(w)$ has halted with \textsc{accept}}
                  \Comment{$w \not\in B$}
                  \State\Return{\textsc{accept}}
              \EndIf
              \Let{i}{i + 1}
          \EndWhile
      \EndFunction
  \end{algorithmic}
\end{algorithm}
\end{figure}

The algorithm \textsf{distinguish} simulates the execution of the Turing
Machines $N, M$ in tandem by running one step in each for ever. Let $C$ be the
language that \textsf{distinguish}$_{N,M}$ computes.

\begin{lemma}
\textsf{distinguish}$_{N,M}$ always halts.
\end{lemma}
\begin{proof}
Assume for contradiction that there were some input $w$ for which
\textsf{distinguish}$_{N,M}$ did not halt for. But then this means that neither
of $N(w), M(w)$ halt, and therefore we have that $w \in A$ and $w \in B$. But
this is a contradiction as, by assumption, we have that $A \cap B = \emptyset$.
\qed
\end{proof}

\begin{lemma}
$A \subseteq C$ and $B \subseteq \overline C$.
\end{lemma}
\begin{proof}
Suppose \textsf{distinguish}$_{N,M}$ is executed with some input $w$ and
halts. Suppose $w \in A$. Then $w \not\in B$, and therefore $w \in \overline B$.
In that case, $M$ will halt with an accepting state and therefore
\textsf{distinguish}$_{N,M}$ will halt and accept as required. The case $w \in
B$ is analogous.
\qed
\end{proof}

\section*{Exercise 10}

Let $K = \{x \in \mathbb{N}: M_x(x)\downarrow\}$ and
$R = \{x \in \mathbb{N}: \text{dom}(\phi_{M_x}) \in \textsc{REC}\}$.

\subsection*{(a) Show that $R \not\subseteq K$}

To show this, it suffices to produce a counter-example of some $x$ such that $x
\in R$ but $x \not\in K$. Indeed let $x$ but the G\"odel encoding of a Turing
Machine that loops for ever. Then clearly $x \not\in K$. But
$\text{dom}(\phi_{M_x}) = \emptyset$ which is trivially computable and so $x
\not\in R$.

\subsection*{(b) Show that $R \cap K \neq \emptyset$}

It suffices to produce a counter-example of some $x$ such that $x \in R \cap K$.
Indeed, let $x$ be the G\"odel encoding of a Turing Machine which has a single
state and halts by immediately accepting. Then clearly $x \in K$. But
$\text{dom}(\phi_{M_x}) = \Sigma^*$ which is trivially computable and so $x \in
R$.

\subsection*{(c) Show that $K \sim \overline K$}

To show this, we note that $K$ is the set of all Turing Machines that terminate
when they are given the encoding of themselves as input, while $\overline K$ is
the set of all Turing Machines that do not terminate when they are given the
encoding of themselves as input. Because $K, \overline K \subseteq \mathbb{N}$,
to show that $K \sim \overline K$, it suffices to show that both $K$ and
$\overline K$ are infinite.

\begin{lemma}
$K$ and $\overline K$ are both infinite.
\end{lemma}
\begin{proof}
$K$ is infinite because there are an infinite number of machines that halt when
executed with their own encoding as input. To see this, take the machine
parameterised by $\kappa \in \mathbb{N}$ that simply counts from $1$ to $\kappa$
and then halts. All of these countably many machines are distinct.

$\overline K$ is infinite because there are an infinite number of machines that
do not halt when executed with their own encoding as input. To see this, take
the machine parameterised by $\kappa \in \mathbb{N}$ that first counts from $1$
to $\kappa$, then resets the counter, and, finally, loops for ever. All of these
countably many machines are also distinct.
\qed
\end{proof}

\section*{Exercise 11}
Let $\mathcal{L} \subseteq \Sigma^*$ and let $N$ be a non-deterministic Turing
Machine deciding $\mathcal{L}$. Show that there is a deterministic Turing
Machine $M$ that decides $\mathcal{L}$.

\paragraph{Solution.}

Given a non-deterministic Turing Machine $N$ executed with some input $w$, we
observe that the computation over an alphabet $\Sigma = \{0, 1\}$ follows a
binary tree. To construct the deterministic machine $M$, we have $M$ simulate
the execution of $N$ in a step-by-step fashion by exploring the computation tree
of $N$ in a breadth-first search fashion. If the machine $N$ terminates, it will
do so at a finite depth on some branch. As $M$ searches using BFS, it will be
able to find the finite depth branch in finite (albeit exponential) time. This
execution strategy is depicted in Figure~\ref{fig.simulation}.

\begin{figure}
    \caption{Simulating a non-deterministic Turing Machine using a
    deterministic Turing Machine by BFS traversal of the execution tree.}
    \centering
    \includegraphics[width=0.5\columnwidth,keepaspectratio]{nd-simulation.png}
    \label{fig.simulation}
\end{figure}

\end{document}
